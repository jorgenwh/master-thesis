\subsubsection{High-Throughput DNA Sequencing} \label{background:biology:high_throughput_dna_sequencing}
\textit{High-throughput sequencing} (HTS), also known as \textit{next-generation sequencing} (NGS), refers to an assortment of recently developed technologies that parallelize the sequencing of DNA fragments to provide unprecedented amounts of genomic sequence data in short amounts of time.
While several such technologies with varying details exist today, they commonly follow a general paradigm. 
%This paradigm consists of performing a template preparation, clonal amplification where they clone pieces of DNA in order to sequence the clones in parallel, and finally cyclical rounds of massively parallel sequencing \cite{hts}.
This paradigm consists of performing a template preparation, clonal amplification where pieces of DNA are cloned in order to sequence the clones in parallel, and finally, cyclical rounds of massively parallel sequencing \cite{hts}.
The resulting DNA sequences produced by HTS technologies are usually referred to simply as (DNA) \textit{reads}.
Such reads are commonly stored as plain text in FASTA or FASTQ files, which can later be used for different kinds of analysis such as genotyping.
Depending on which HTS technology is used, one can expect read lengths ranging from as low as 150 bases using \textit{Illumina} technologies, referred to as \textit{short reads} \cite{illumina_read_length}, all the way up to 15-20 thousand bases using \textit{Pacific Biosciences} (PacBio) technologies, referred to as \textit{long reads} \cite{hts2}.
There are three important factors to consider when determining which HTS technology to use for a given purpose: 1) the read lengths produced, 2) the average probability for each base being erroneous, usually referred to as the error rate, and 3) the cost of sequencing given the technology, which can potentially limit how much data one may be able to produce.

\begin{figure}[H]
\begin{center}
\small{example.fa}
\end{center}
\begin{lstlisting}[style=vcf]
>read 0
ACGTATGCGGCGGGGCGCGATTATTCGTTGCGTATGC
>read 1
ACACGTCGTGCGTAGCGTGTCAGTCACAGTAAACAAA
>read 2
CGTTGCCATCAACGGCTGTGCACGATTGGGGGCGCGC
...
\end{lstlisting}
\caption{
  An illustration of how sequenced DNA reads are stored as plain text in a FASTA (.fa) file.
  Before each read, a header file beginning with the character ">" may provide information about the read.
}
\label{background:biology:high_throughput_dna_sequencing:figures:fasta}
\end{figure}

When performing DNA sequencing, we have no knowledge of where a read originates from in the original genome.
This issue, in addition to the fact that some bases are erroneously sequenced, introduces uncertainties when trying to predict features about an individual's genome based on the sequenced reads.
Many techniques rely on \textit{aligning} (or \textit{mapping}) sequenced reads to a reference genome sequence in order to better predict where in the genome the read originates from.
To gain better confidence and local support along these alignments, it is common to sequence an individual's genome at higher coverages, so that aligned reads may overlap and correspondance can be asserted.
The term sequencing \textit{coverage} is used as an indication of how many times each nucleotide in a genome has been sequenced on average.
E.g. 15x coverage means that every nucleotide in the genome has been sequenced, on average, 15 times.
