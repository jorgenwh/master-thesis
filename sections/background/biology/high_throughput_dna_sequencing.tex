\subsubsection{High-Throughput DNA sequencing} \label{background:biology:high_throughput_dna_sequencing}
\textit{High-throughput sequencing} (HTS), also known as \textit{next-generation sequencing} (NGS), refers to an assortment of recently developed technologies that parellalize the sequencing of DNA fragments to provide unprecedented amounts of genomic data in short amounts of time.
While several such technologies with varying details exist today, they commonly follow a general paradigm of performing a template preparation, clonal amplification where they clone pieces of DNA in order to sequence the clones in parallel, and finally cyclical rounds of massively parallel sequencing \cite{hts}.
The resulting DNA sequences produced by HTS technologies are usually referred to simply as (DNA) \textit{reads}.
Such reads are commonly stored as plain text in FASTA or FASTQ files, which can later be used for different kinds of analysis such as genotyping.
Depending on which HTS technology is used, one can expect read lengths ranging from as low as 150 bases using \textit{Illumina} technologies, referred to as \textit{short reads} \cite{illumina_read_length}, all the way up to 15-20 thousand bases using \textit{Pacific Biosciences} (PacBio) technologies, referred to as \textit{long reads} \cite{hts2}.
Three factors are important to consider when determining which HTS technology to use for a given purpose: 1) the read lengths produced, 2) the average probability for each base being erroneous, usually referred to as the error rate, and 3) the cost of sequencing given the technology, which can potentially limit how much data one may be able to produce.

\begin{figure}[H]
\begin{center}
\small{example.fa}
\end{center}
\begin{lstlisting}[style=vcf]
>0
ACGTATGCGGCGGGGCGCGATTATTCGTTGCGTATGC
>1
ACACGTCGTGCGTAGCGTGTCAGTCACAGTAAACAAA
...
\end{lstlisting}
\caption{
  An illustration of how sequenced DNA reads are stored as plain text in a FASTA (.fa) file.
  Before each read, a header file beginning with the character ">" may provide information about the read.
}
\label{background:biology:high_throughput_dna_sequencing:figures:fasta}
\end{figure}
