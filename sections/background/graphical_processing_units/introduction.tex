\textit{Graphical Processing Units} (GPUs) are massively parallel processing units designed for high-throughput parallel computations.
This is as opposed to \textit{Central Processing Units} (CPUs), which are designed to quickly perform many serial computations.
GPUs were originally developed to accelerate computations performed on images, a highly parallel task where it is commonplace to have millions of relatively small independent computations that must be performed quickly in a single memory buffer.
Although GPUs have mainly been used for graphical computations, they have in recent years been adopted in other areas as well with the introduction of the \textit{General Purpose Graphical Processing Unit} (GPGPU) (\textbf{GPGPU cite}). 
The concept of the GPGPU is to use a GPU, which is designed for computer graphics, to perform computations in other domains where CPUs are typically used.
Fields such as artificial intelligence (\textbf{AI accelerated by GPU cite}) and the broader scientific computing have enjoyed great utility from GPUs, using them to accelerate embarassingly parallel problems, \textit{e}.\textit{g}., matrix operations.
Despite being similar in power consumption, a GPU can provide much higher instruction throughput and memory bandwidth compared to its CPU competitors.
Such capability advantages exist in GPUs because they were specifically designed to perform well with regards to these dimensions.
Modern GPUs can in effect be considered to be massive \textit{Single Instruction Multiple Data} (SIMD) machines. Strict flow control is therefore important; The same set of instructions should run in the same order for maximum utilization of the GPU's capability.
Furthermore, since GPUs are commonly a discrete compute unit from a computer's CPU, it contains its own \textit{Random Access Memory} (RAM) unit, requiring data to be copied from the ...


As of 2023, Nvidia control the vast majority of the GPU market share, with only \textit{Advanced Micro Devices} (AMD) and Intel as current serious competitors (\textbf{try to find a serious cite}).
Furthermore, Nvidia GPUs with their CUDA programming model is considered to be the standard for scientific computing today (\textbf{cite}).
Although most of the GPUs manufactured by different GPU manufacturing companies are very similar in both architecture and compute models, the term \textit{GPU} will for the remainder of this thesis specifically refer to Nvidia GPUs, as the work presented in this thesis was developed and tested using only Nvidia GPUs.

