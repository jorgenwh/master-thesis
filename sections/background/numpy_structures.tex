\subsection{NumPy Structures} \label{background:npstructures}
NumPy Structures (npstructures) is a Python package built on top of NumPy that provides data structures with NumPy-like features to augment the NumPy library \cite{npstructures}.

A central feature in NumPy Structures is its ragged array data structure, a two-dimensional array with differing column lengths that provides NumPy-like behaviour and performance.
\begin{figure}[H]
\begin{lstlisting}[language=Python,style=console]
>>> from npstructures import RaggedArray
>>> data = [0, 1, 2, 3, 4, 5, 6, 7, 8]
>>> column_lengths = [4, 2, 3]
>>> ra = RaggedArray(data, column_lengths)
>>> ra
ragged_array([0, 1, 2, 3]
[4, 5]
[6, 7, 8])
>>> ra.ravel()
array([0, 1, 2, 3, 4, 5, 6, 7, 8])
>>> type(ra.ravel())
<class 'numpy.ndarray'>
\end{lstlisting}
\caption{
  A simple illustration of how NumPy Structures' data structures can be used directly in Python as drop-in augmentations to the NumPy library.
}
\label{background:npstructures:figure}
\end{figure}

NumPy Structures also provides a hash table built on top of the ragged array data structure.
This hash table is designed to give dictionary-like behaviour for NumPy-arrays, meaning chunks of key-value pairs can be operated on at once using fast NumPy array routines.
