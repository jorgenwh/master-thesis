\subsection{Drawbacks of Graphical Processing Units}
While GPUs can provide excellent acceleration for many problems in scientific computing, they do come with some notable caveats.

The reason why GPUs are powerful when it comes to accelerating certain parallel programs, is because they are designed for such problems.
GPUs are highly specialized compute accelerators that perform poorly when applied to any problem that does not fit its compute architecture.
Additionally, today's GPUs are expensive and less accessible than traditional CPUs.
Because of the CPU's flexibility in regards to the problems it can solve, CPUs exist in most - if not all - existing computers today.
GPUs, however, are less common, partly because of cost.

Furthermore, GPU programming is its own discipline, as the programming models and paradigms used when developing GPU programs are quite different from the typical sequential programs written for the CPU.
This leads to a higher bar for entry when it comes to developing effective GPU programs for difficult problems, as fewer expert programmers have the knowledge and training to implement such solutions.

