In section \ref{methods} we demonstrated three distinct methods of GPU accelerating existing Python code that uses NumPy.
We used a combination of these methods to implement GPU acceleration for KAGE, resulting in a new and improved version: GKAGE.
In section \ref{results} we showed that GKAGE achieves a 5-11 times speedup compared to KAGE, depending on the type of hardware it is benchmarked on.

In this section we discuss some topics relevant to the project presented in thesis.
This will include:
\begin{compactitem}
  \item
    \textit{\textbf{Not using Gerbil for kmer counting}}: why did we implement our own GPU accelerated \textit{k}mer counting tool when a previous \textit{k}mer counting tool, Gerbil \cite{gerbil}, already had GPU acceleration support?
  \item
    \textit{\textbf{Advantages and drawbacks of the GPU acceleration methods used}}: Why do several possible GPU acceleration libraries and methods exist and why is there no one-size-fits-all solution?
  \item
    \textit{\textbf{Drawbacks of GPUs}}: what challenges do GPUs introduce?
  \item
    \textit{\textbf{Significance of GKAGE}}: what do GKAGE and its results mean for bioinformatics?
  \item
    \textit{\textbf{Future work}}: what are interesting and perhaps obvious improvements that can still be integrated into GKAGE?
\end{compactitem}
