\subsection{Thesis Goals} \label{introduction:thesis_goals}
This thesis has two main goals.
One of the goals is to explore whether state-of-the-art genotyping can be sped up in any significant way by utilizing GPUs.
More specifically, this thesis will investigate whether alignment-free genotyping, which presently is significantly faster compared to alignment-based genotyping, can be sped up by utilizing the GPU.
To investigate this, we will attempt to integrate GPU accelerated functionality into a base alignment-free genotyper, KAGE, which is presently the fastest known genotyping tool that also yields competitive results.
The base genotyper, KAGE, is implemented in Python and relies heavily on the array-library NumPy for performance.
Thus, there are several possible avenues for integrating GPU support. 
Low level implementations in C++ using the CUDA framework provides great granularity and control, but require deep knowledge and understanding of hardware, the C++ programming language and CUDA.
Alternative methods may be to use existing Python packages that provide out-of-the-box GPU accelerated functionality.
This leads us to the second goal of this thesis - to investigate and experiment with different ways of GPU accelerating existing Python code that relies on array-programming libraries such as NumPy, and to reveal advantages and drawbacks of each method.
