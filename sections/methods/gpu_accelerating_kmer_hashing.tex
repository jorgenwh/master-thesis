\subsection{GPU Accelerating \textit{k}mer Hashing} \label{methods:gpu_accelerating_kmer_hashing}
The \textit{k}mer hashing component of KAGE is responsible for reading genomic reads from FASTA, FASTQ or other file types, encoding the reads as 2-bit encoded data and finally extracting (or hashing) all valid \textit{k}mers from the 2-bit encoded genomic reads data.
In KAGE, the final product yielded by this component is a 64-bit unsigned integer array where each element is a 2-bit encoded 31-mer represented in the right-most 62 bits of the integer.
Since the number of valid \textit{k}mers in a typical FASTA or FASTQ from sequencing a human genome is extremely vast, the file is usually read, encoded and hashes in chunks to alleviate memory consumption.
Functionality for this exists in the BioNumPy Python package.
BioNumPy implements this using NumPy and some parts of NumPy Structures, resulting in an efficient solution that relies on NumPy's fast array operations for its performance.
