The following chapter will describe the methods used and implementation details of how we created GPU support for KAGE.

%The KAGE genotyping pipeline consists of several steps. 
%An abstract overview of the KAGE pipeline can be represented by the following two steps:
%\begin{compactenum}
%  \item
%    Read the input fasta file, extract and encode all valid \textit{k}mers and count the observed \textit{k}mer frequencies of the predetermined set.
%  \item
%    Determine the genotypes based on the observed \textit{k}mer frequencies.
%\end{compactenum}

%\begin{figure}[!ht]\label{figure:KAGE-pipeline}
%\begin{center}
%\begin{tikzpicture}
  % draw kmer counting box
%  \node at(-2,0)[rectangle,draw,rounded corners,minimum height=1.75cm,minimum width=3cm](kmercounting){\textit{k}mer counting};
  % draw genotype box
%  \node at(2,0)[rectangle,draw,rounded corners,minimum height=1.75cm,minimum width=3cm](genotype){genotype};
  % draw "KAGE" text
%  \node at(0,2){KAGE};
  % draw KAGE box
%  \node at(0,0)[rectangle,draw,rounded corners,minimum width=9cm,minimum height=3cm](kage){};
  % draw arrow from kmer counting to genotype
%  \draw [-stealth](kmercounting) -- (genotype);
%\end{tikzpicture}
%\caption{...}
%\end{center}
%\end{figure}
