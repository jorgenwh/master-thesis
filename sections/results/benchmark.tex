\subsection{Benchmark} \label{results:benchmark}
In order to benchmark the effectiveness of the added GPU acceleration in GKAGE, we decided to benchmark GKAGE against KAGE to account for the speedup.
The choice of only benchmarking GKAGE against KAGE was made based on the fact that KAGE recently showed that it was an order of magnitude faster than any other known genotyper \cite{kage}.

\subsubsection{Snakemake pipeline}
In order to adequately benchmark GKAGE, we set up a Snakemake pipeline that runs both KAGE and GKAGE on a full human genome and records the runtimes while also checking that the results of both processes are identical.
This Snakemake pipeline can be found at \url{https://github.com/kage-genotyper/GKAGE-benchmarking}.

\begin{table}[H]
\begin{tabular}{lllll}
\cline{1-3}
\multicolumn{1}{|l|}{\textbf{System}} & \multicolumn{1}{l|}{\textbf{CPU}}                  & \multicolumn{1}{l|}{\textbf{GPU}}                   &  \\ \cline{1-3}
\multicolumn{1}{|l|}{1: High-end server} & \multicolumn{1}{l|}{AMD EPYC 7742}        & \multicolumn{1}{l|}{Nvidia Tesla V100}     &  \\ \cline{1-3}
\multicolumn{1}{|l|}{2: Consumer desktop} & \multicolumn{1}{l|}{Intel Core i5-11400F} & \multicolumn{1}{l|}{Nvidia GTX 1660 SUPER} &  \\ \cline{1-3}
\end{tabular}
\caption{
  The two systems used to benchmark GKAGE against KAGE to account for the speedup achieved through GPU acceleration.
  \textbf{System 1} is a high-end compute server with top-of-the-line hardware.
  \textbf{System 2} is a consumer grade desktop gaming computer.
}
\label{results:gkage:tables:systems}
\end{table}
