\subsection{GKAGE} \label{results:gkage}
After having explored several possible methods for GPU accelerating the existing NumPy-based Python code in KAGE, we integrated the GPU accelerated functionality into KAGE.
We called the resulting program GKAGE (GPU KAGE).
While it can seem like GKAGE and KAGE are two separate programs, GKAGE is ran today by using a -g flag to utilize the GPU accelerated functionality of GKAGE.

\begin{table}[H]
\begin{tabular}{lllll}
\cline{1-4}
\multicolumn{1}{|l|}{\textbf{System}} & \multicolumn{1}{l|}{\textbf{Description}}               & \multicolumn{1}{l|}{\textbf{CPU}}                  & \multicolumn{1}{l|}{\textbf{GPU}}                   &  \\ \cline{1-4}
\multicolumn{1}{|l|}{1}      & \multicolumn{1}{l|}{High-end compute server}   & \multicolumn{1}{l|}{AMD EPYC 7742}        & \multicolumn{1}{l|}{Nvidia Tesla V100}     &  \\ \cline{1-4}
\multicolumn{1}{|l|}{2}      & \multicolumn{1}{l|}{Consumer desktop} & \multicolumn{1}{l|}{Intel Core i5-11400F} & \multicolumn{1}{l|}{Nvidia GTX 1660 SUPER} &  \\ \cline{1-4}
\end{tabular}
\caption{
  The two systems used to benchmark GKAGE against KAGE to account for the speedup achieved through GPU acceleration.
  \textbf{System 1} is a high-end compute server with top-of-the-line hardware.
  \textbf{System 2} is a consumer grade desktop gaming computer.
}
\label{results:gkage:tables:systems}
\end{table}
